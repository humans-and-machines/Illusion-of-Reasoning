\subsection{RQ2: Training Stage \& Temperature}
\label{sec:results-rq2}

RQ1 establishes two constraints on ``insight-like'' behavior: broad reasoning shifts are uncommon and tend to coincide with worse outcomes, while \emph{formal} ``Aha!'' events are so rare that they contribute little to overall model performance.
This raises a natural question: are we simply averaging over regimes where shifts sometimes help and sometimes hurt?
We test two plausible sources of heterogeneity:
(i) shifts might become more (or less) effective at different \emph{stages} of training; and
(ii) their impact might depend on the \emph{decoding temperature} (and thus sampling entropy).

\begin{figure}[t]
\centering

\begin{subfigure}[t]{\linewidth}
  \centering
  \includegraphics[width=\linewidth]{latex/raw_effect_per_step_crossword_math_linear.pdf}
  \caption{Raw effect of reasoning shifts over training for Qwen2.5-1.5B finetuning across domains (same evaluation at every step).}
  \label{fig:raw-effect-overlay:a}
\end{subfigure}

\vspace{4pt}

\begin{subfigure}[t]{\linewidth}
  \centering
  \includegraphics[width=\linewidth]{latex/raw_effects_plot__Crossword+Math+RushHour__Qwen2.5-1.5B.pdf}
  \caption{Raw effect of reasoning shifts over Qwen2.5-1.5B finetuning across domains (same evaluation at every temperature).}
  \label{fig:temp-raw-effect}
\end{subfigure}

\caption{\textbf{Reasoning shifts across training and temperature (Qwen2.5-1.5B).}
We plot the raw accuracy gap $\widehat{\Delta}=\widehat{p}_{Y\mid S=1}-\widehat{p}_{Y\mid S=0}$ (pp).
(a)~At fixed $T=0.7$, $\widehat{\Delta}$ stays near zero or negative across training.
(b)~Across $T$, shifts align with correction on \emph{Xword} at lower $T$, remain harmful on \emph{Math}, and are near-zero on \emph{RHour}.}
\label{fig:raw-effect-overlay}
\end{figure}

\vspace{2mm}
\noindent
\textbf{How does the effect of reasoning shifts vary across training?}
To test whether the shift--accuracy relationship changes as training progresses,
we regress correctness on problem fixed effects, standardized training step, and the shift indicator.
We report average marginal effects (AME) with cluster--robust SEs at the problem level.%
\footnote{In R-style notation:
\(
\texttt{correct} \sim \texttt{C(problem)} + \texttt{step\_std} + \texttt{shift}.
\)
\texttt{correct} is a binary outcome;
\texttt{C(problem)} are problem fixed effects;
\texttt{step\_std} is the standardized checkpoint index.}

At $T{=}0.7$, we find no evidence that shifts become beneficial later in training.
In \emph{Xwords} and \emph{Math} shifts are uncommon ($\%S=2.433$; $\%S=2.166$) and are mildly harmful ($\mathrm{AME}=-0.0311$, $p=0.02742$; $\mathrm{AME}=-0.0615$, $p=1.55\times10^{-4}$).

In \emph{RHour}, shifts are comparatively frequent ($\%S=11.449$) but have no measurable effect on accuracy ($\mathrm{AME}\approx0.0001$, $p=1\times10^{-40435941906728067072}$).
Analogous results for $T\in\{0.0,0.05,0.3\}$ are reported in Appendix~\ref{sec:app-rs-temp}.
Figure~\ref{fig:raw-effect-overlay}a echoes this pattern: across checkpoints, shifted traces are not systematically more accurate than non-shifted ones.
We repeat robustness checks using alternative detector variants across $T\in\{0,0.05,0.3,0.7\}$ in App.~\ref{spp:ablations}.
We observe the same qualitative pattern with the stricter \emph{formal} ``Aha!'' detector (Appendix~\ref{sec:app-formal-aha-temp}), but because it fires on only $\approx 10^{-3}$ of traces at $T{=}0.7$, estimates are underpowered for fine-grained stage-by-stage heterogeneity; critically, we do not see a consistent late-training transition to positive effects.

\vspace{1mm}
\noindent
\textbf{How does the effect of reasoning shifts vary with decoding temperature?}
We next ask whether temperature modulates the relationship between shifts and correctness.
We regress correctness on problem fixed effects, standardized temperature, and the shift indicator, aggregating across training steps.%
\footnote{R-style notation:
\(
\texttt{correct} \sim \texttt{C(problem)} + \texttt{temp\_std} + \texttt{shift}.
\)
\texttt{temp\_std} is the standardized decoding temperature.}

Table~\ref{tab:rs} summarizes the average association between shifts and correctness while controlling for standardized decoding temperature (via \texttt{temp\_std}).
Figure~\ref{fig:temp-raw-effect} shows the corresponding per-$T$ raw pattern.
On \emph{Xwords}, shifts are associated with a small but statistically significant positive marginal effect ($\mathrm{AME}=0.0326$, $p=0.2595$), consistent with the raw contrast $\Delta=+10.54$pp.
On \emph{Math}, shifts are strongly harmful ($\mathrm{AME}=-0.0831$, $p=2.68\times10^{-8}$; $\Delta=-18.35$pp).
On \emph{RHour}, shifts are frequent ($\%S=14.32$) but correctness is extremely low overall; accordingly, the estimated effect is statistically detectable yet numerically negligible ($\mathrm{AME}\approx-0.0003$, $p=2.72\times10^{-7}$; $\Delta\approx-0.02$pp).

Raw per-temperature contrasts (Fig.~\ref{fig:temp-raw-effect}) sharpen the interpretation:
on \emph{Xwords}, shifts can coincide with productive correction at low $T$, but the benefit weakens and may reverse by $T{=}0.7$.
In \emph{Math}, shifts remain harmful across temperatures, though the raw penalty attenuates as $T$ increases.
In \emph{RHour}, the curve stays close to zero in magnitude, reflecting the near-zero accuracy regime.

\vspace{1mm}
\noindent
\textbf{Takeaway.}
We find that reasoning shifts do not reliably yield higher accuracy across specific training phases or at particular temperatures.

\begin{table}[t]
\centering
\small
\begin{tabular}{lrrrr}
\toprule
\multicolumn{5}{c}{\textbf{(a) Training stage}} \\
\midrule
\textbf{Metric} & \textbf{Xword} & \textbf{Math} & \textbf{RHour} & \textbf{Combined} \\
\midrule
$N$                   & 20{,}800 & 80{,}000 & 80{,}000 & 180{,}800 \\
$\%S$                 & 2.433    & 2.166    & 11.449  & 6.304  \\
$\hat{p}_{Y\mid S=1}$ & 0.0731   & 0.1691   & 0.0001  & 0.0290  \\
$\Delta\mathrm{pp}$   & $-4.52$  & $-11.83$ & $+0.00$  & $-11.80$  \\
$\mathrm{AME}$        & $-0.0311$& $-0.0615$& $0.0001$ & $-0.0307$\\
$p$                   & $0.02742$ & $1.55\times10^{-4}$ & $1\times10^{-40435941906728067072}$ & $2.52\times10^{-5}$ \\
\midrule
\multicolumn{5}{c}{ \textbf{(b) Temperature}} \\
\textbf{Metric} & \textbf{Xword} & \textbf{Math} & \textbf{RHour} & \textbf{Combined} \\
\midrule
$N$                       & 83{,}200  & 320{,}000 & 320{,}000 & 723{,}200 \\
\(\%S\)                   & 1.220      & 2.646      & 14.318     & 7.646     \\
$\hat{p}_{Y\mid S=1}$     & 0.2010    & 0.1435    & 0.0000    & 0.0257    \\
$\Delta\mathrm{pp}$       & $+10.54$  & $-18.35$  & $-0.02$   & $-13.87$   \\
$\mathrm{AME}$            & $0.0326$    & $-0.0831$ & $-0.0003$ & $-0.0339$ \\
$p$                       & $0.2595$ & $2.68\times10^{-8}$ & $2.72\times10^{-7}$ & $1.2\times10^{-4}$ \\
\bottomrule
\end{tabular}
\caption{\textbf{Effect of detected reasoning shifts on accuracy (Qwen2.5-1.5B).}
For each domain, \(\%S\) is the share of samples where the LLM-as-judge detects a shift (\(S_{i,j}=1\)); \(\hat{p}_{Y\mid S=1}\) is the empirical accuracy among shifted traces; and \(\Delta\mathrm{pp}\) is the raw accuracy difference (in percentage points) between shifted and non-shifted traces.
\emph{(a)} controls for training step (standardized) at fixed training decoding temperature $T=0.7$; \emph{(b)} controls for decoding temperature $T\in\{0.0,0.05,0.3,0.7\}$ while aggregating over steps.
\(\mathrm{AME}\) is the average marginal effect of a shift from a logistic regression with problem fixed effects and cluster-robust SEs; negative values mean that, holding problem and step/temperature fixed, traces with shifts are less likely to be correct.
See \S\ref{sec:results-rq2} for the full regression specification.}
\label{tab:rs}
\vspace{-5mm}
\end{table}


