\subsection{RQ3: Reasoning Shifts \& Uncertainty}
\label{sec:rq3-uncertainty}

The results above (particularly \textit{Xwords}, see Fig.~\ref{fig:temp-raw-effect}) suggest that decoding temperature may modulate the effect of reasoning shifts: at low $T$ they sometimes align with productive corrections, while at higher $T$ they resemble noise. Because temperature primarily alters sampling entropy rather than the model's underlying reasoning process~\citep{hinton2015distilling,holtzman2019degeneration}, this points to a link between shifts and internal uncertainty. We thus ask whether, under high-uncertainty regimens, reasoning shifts are more frequent or become more helpful.

\vspace{1mm}
\noindent
\textbf{Are reasoning shifts more likely under high uncertainty?}
To test whether shifts preferentially occur when the model is uncertain, we relate each trace's reasoning shift indicator to its sequence-level entropy. We pool traces across all decoding temperatures and training checkpoints, and fit a logistic regression of shift prevalence on standardized entropy with problem fixed effects and cluster-robust SEs (clustered by problem).%
\footnote{In R-style notation:
\(
\texttt{shift} \sim \texttt{C(problem)} + \texttt{std\_entropy}.
\)
Here \texttt{shift} is a binary indicator for reasoning shift, \texttt{C(problem)} denotes problem fixed effects, and \texttt{std\_entropy} is the within-domain $z$-scored pass-1 sequence entropy. We estimate a Binomial(logit) GLM with cluster-robust SEs at the problem level.}

Pooling \emph{all} traces across domains (\textit{Xwords}, \textit{Math}, \textit{RHour}), we find that higher entropy is associated with a higher probability of a shift on average: a $+1$ SD increase in entropy corresponds to an odds ratio of $\approx$0.77$\times$ (log-odds $\beta=-0.258$, $\mathrm{SE}=0.1427$, $p=0.07034$; 95\% CI OR [0.58, 1.02]; $N=723{,}200$). However, this aggregate effect masks domain heterogeneity: the entropy--shift association is positive in \textit{Xwords} (OR$\approx$2.05$\times$) and \textit{RHour} (OR$\approx$1.19$\times$), but negative in \textit{Math} (OR$\approx$0.58$\times$).
One possible intuition is that in \emph{Math}, higher-entropy generations more often reflect diffuse exploration or verbose ``flailing'' rather than a discrete mid-trace pivot, so shifts concentrate in comparatively lower-entropy traces.
Across domains, shifts remain rare in absolute terms ($\approx$0.03\%--0.26\%), so these effects primarily indicate \emph{where the rare shifts concentrate} in the entropy distribution, rather than implying that shifts become common.

\vspace{1.5mm}
\noindent
\textbf{Do reasoning shifts improve performance under high uncertainty?}
A natural hypothesis is that when the model is uncertain, a mid-trace pivot might reflect productive self-correction. We test this by stratifying traces into \emph{high-entropy} instances (top 20\% within domain) and \emph{low-entropy} instances (bottom 80\%), using a fixed entropy threshold per domain. Within each stratum, we estimate the effect of a shift on correctness using a logistic regression with problem fixed effects and controls for continuous entropy and training step, and report the shift coefficient alongside the raw accuracy difference between shifted and non-shifted traces.%
\footnote{Within each domain, we split at the 80th percentile of sequence entropy and fit a Binomial(logit) GLM predicting \texttt{correct} from \texttt{shift} with problem fixed effects and covariates. We report both regression and raw contrasts for interpretability.}

Table~\ref{tab:rq3-shift-entropy-strata} shows that shifts do \emph{not} become reliably beneficial in the high-entropy regime. In \emph{Math}, shifts remain harmful even among high-entropy traces (raw $\Delta=-7.40$pp) and are substantially more harmful in the low-entropy majority (raw $\Delta=-22.88$pp). In \emph{Xwords}, the point estimate in the high-entropy stratum is near zero (raw $\Delta=+0.63$pp), but shifts are so rare that estimates are noisy and not statistically distinguishable from zero. In \emph{RHour}, accuracy is near-zero throughout, so estimated effects are statistically detectable due to sample size but negligible in magnitude.

\begin{table}[t]
  \centering
  \small
  \setlength{\tabcolsep}{4pt}
  \renewcommand{\arraystretch}{1.05}
  \begin{tabular*}{\columnwidth}{@{\extracolsep{\fill}} l r r r r @{}}
    \toprule
    \textbf{Metric} & \textbf{Xword} & \textbf{Math} & \textbf{RHour} & \textbf{Combined} \\
    \midrule
    \multicolumn{5}{c}{\textbf{All traces}} \\
    \midrule
    $N$                       & 83{,}200  & 320{,}000 & 320{,}000 & 723{,}200 \\
    $\Delta$ (pp)             & $-6.24$   & $-19.78$  & $-0.02$   & $-5.79$   \\
    $\mathrm{coef(shift)}$    & $-1.49$   & $-1.11$   & $-22.76$  & $-1.10$  \\
    $p$                       & $0.123$     & $2.25\times10^{-7}$ & $4.47\times10^{-103249}$ & $2.52\times10^{-7}$ \\
    \midrule
    \multicolumn{5}{c}{\textbf{High entropy (top 20\%)}} \\
    \midrule
    $N$                       & 16{,}640  & 64{,}000 & 64{,}000 & 144{,}640 \\
    $\Delta$ (pp)             & $+0.63$   & $-7.40$  & $-0.03$   & $-3.79$   \\
    $\mathrm{coef(shift)}$    & $-0.04$   & $-0.28$   & $-22.48$  & $-0.24$  \\
    $p$                       & $0.904$     & $0.739$ & $1.84\times10^{-54405}$ & $0.726$ \\
    \midrule
    \multicolumn{5}{c}{\textbf{Low entropy (bottom 80\%)}} \\
    \midrule
    $N$                       & 66{,}560  & 256{,}000 & 256{,}000 & 578{,}560 \\
    $\Delta$ (pp)             & $-10.00$   & $-22.88$  & $-0.02$   & $-6.83$   \\
    $\mathrm{coef(shift)}$    & $-28.83$   & $-1.14$   & $-22.90$  & $-1.15$  \\
    $p$                       & $1.33\times10^{-46249}$     & $4.96\times10^{-7}$ & $2.15\times10^{-114331}$ & $4.86\times10^{-7}$ \\
    \bottomrule
  \end{tabular*}
  \caption{\textbf{Do spontaneous reasoning shifts help under high uncertainty?}
  We stratify traces within each domain by sequence entropy (high = top 20\% at the within-domain 80th percentile; low = bottom 80\%), and compare shifted vs.\ non-shifted traces. $\Delta$ (pp) is the raw accuracy difference
  $\hat p(\checkmark\!\mid\!S{=}1) - \hat p(\checkmark\!\mid\!S{=}0)$. $\mathrm{coef(shift)}$ and $p$ report the shift coefficient and $p$-value from a logistic regression with problem fixed effects and covariates. Across domains, shifts do not become reliably beneficial in the high-entropy regime.}
  \label{tab:rq3-shift-entropy-strata}
\end{table}


\vspace{1mm}
\noindent
\textbf{Can artificially triggered reasoning shifts improve performance?}
The negative results above suggest that \emph{spontaneous} shifts are not a dependable self-correction mechanism, even when the model is uncertain.
We therefore test an \emph{extrinsically triggered} ``forced Aha'' intervention: for each prompt we generate a baseline completion (Pass~1), then re-query the model under identical decoding settings while appending a reconsideration cue (Pass~2), and compare paired correctness outcomes.
Here Pass~1 is the model's initial answer, while Pass~2 appends a short instruction that explicitly asks the model to re-check and revise if needed (the same cue across all domains; see App.~\ref{app:uncertainty-interventions} for the exact wording).

Table~\ref{tab:rq3-forced-aha} reports sample-level paired results aggregated across checkpoints and decoding temperatures, reported separately by domain. Forced reconsideration yields a large gain on \emph{Math} ($$0.322\!\rightarrow\!0.406$$; $+8.41$pp) and a small gain on Xwords ($+0.45$pp), while remaining negligible in absolute terms on RHour ($+0.013$pp) due to its near-zero base rate.
Importantly, the paired ``win'' counts show that improvements dominate backslides in \emph{Math} (50{,}574 wrong$\rightarrow$right vs. 23{,}500 right$\rightarrow$wrong), indicating that the effect is not merely random flipping.
In contrast, \emph{Xwords} shows near-balanced wins and losses (5{,}380 vs. 5{,}004), consistent with a much smaller net gain. Across domains, triggered reconsideration improves accuracy (Table~\ref{tab:rq3-forced-aha}). In \emph{Math}, these gains are amplified on high-entropy instances, consistent with uncertainty serving as a useful gate for reflection (App.~\ref{app:uncertainty-interventions}, Table~\ref{tab:cue-regressions}).

\vspace{1mm}
\noindent
\textbf{Takeaway.} Reasoning shifts are a low-base-rate event that concentrates in higher-entropy (more uncertain) generations on average, though the direction and strength of this association varies by domain.
Conditioning on uncertainty does not reveal a ``hidden regime'' where spontaneous shifts reliably help. If anything, the primary pattern is that shifts are least harmful in the high-entropy tail, but they do not turn into a consistent mechanism for improving correctness.
Artificially triggering reasoning shifts yields performance gains, particularly in \emph{Math} under high-entropy.

\begin{table}[t]
  \centering
  \small
  \setlength{\tabcolsep}{4pt}
  \renewcommand{\arraystretch}{1.05}
  \begin{tabular*}{\columnwidth}{@{\extracolsep{\fill}} l r r r r @{}}
    \toprule
    \textbf{Metric} & \textbf{Xword} & \textbf{Math} & \textbf{RHour} & \textbf{Combined} \\
    \midrule
    $N$ (paired samples)      & 83{,}240  & 321{,}938  & 322{,}050 & 727{,}228 \\
    $\hat p_{\text{P1}}$      & 0.0970    & 0.3221    & 0.000233 & 0.1538 \\
    $\hat p_{\text{P2}}$      & 0.1015    & 0.4062    & 0.000363 & 0.1916 \\
    $\Delta$ (pp)             & $+0.45$   & $+8.41$   & $+0.01$ & $+3.78$ \\
    wins (P2 $\uparrow$)      & 5{,}380   & 50{,}574  & 100 & 56{,}054 \\
    wins (P1 $\uparrow$)      & 5{,}004   & 23{,}500  & 58 & 28{,}562 \\
    \bottomrule
  \end{tabular*}
  \caption{\textbf{Forced ``Aha'' (triggered reconsideration), sample-level results.}
  We compare paired outcomes between a baseline generation (Pass~1) and a second generation with an appended reconsideration cue (Pass~2).
  $\hat p_{\text{P1}}$ and $\hat p_{\text{P2}}$ denote accuracies in each pass; $\Delta$ (pp) is the percentage-point gain.}
  \label{tab:rq3-forced-aha}
  \vspace{-4mm}
\end{table}


